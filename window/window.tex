
\documentclass[12pt]{article}
\begin{document}

\tableofcontents

\section{Notation}

\subsection{Coordinate Systems [$CS$]}

A Coordinate system (often referred to as $CS$ ) is specified using a point (called its origin and is denoted by $CS_o$) and two unit axes ($CS_1$ and $CS_2$). 


\subsection{Unit}

The length of a unit axis of a $CS$ is refered to as the $CS$'s \textbf{unit}.


\subsection{Points}

All points are assumed to be two dimensionsional.

A point is written as $p_{CS}$, with $CS$ representing the specific coordinate system. A point measured in the $millimeter$ coordinate system for instance, should be written as  $p_{mm}$.

Formally a point is defined as $p = p_{CS} := \{(x, y) \in \mathbb{R}^2\}$, unless explicitly stated otherwise. Only x or y can be referred to using the set $p_{\{x, y\}}$ or $p_{CS_{\{x, y\}}}$.

\section{GLFW}

https://www.glfw.org

As the Physimos window depends on GLFW for cross platform OS interaction, some basic GLFW coverage is necessary here.

\subsection{Units}

The primary unit of the GLFW library is (virtual) screen coordinates, which we will refere to as $sc$.


\subsection{Coordinate systems}

Two coordinate systems are used. They differ only by a translation. 

\textit{Virtual Screen} : measured from the top left corner of the primary monitor. A point in virtual screen coordinates will be written as $p_{sc}$.

\textit{Content Area} : 




\section{Physimos Window [PW]}

\subsection{Coordinate Systems}


% TABLE
\begin{center}
% \begin{tabular}[t]{ | m{5em} | m{1cm}| m{1cm} | }
% \begin{tabular}[b]{ | m{5em} || m{2cm} | m{1cm} | } 
\begin{tabular}{|| c | c | c | c ||}
    \hline
    Name & Symbol & Unit & Matrix \\ 
    \hline
    \hline
    Input & i & \textit{sc} &   \[ p_{target} = M_{i/target} p_i
                        \] \\ 
    \hline
    Sanity & s & \textit{sc} &   \[ M_{i/s} = 
                            \begin{pmatrix}
                            1 & 0 & 0 \\
                            0 & -1 & h_w \\
                            0 & 0 & 1
                            \end{pmatrix}
                        \] \\ 
    \hline
    Normalized & n & nn & \[ M_{n/s} = 
                            \begin{pmatrix}
                            \frac{1}{w_w} & 0  \\
                            0 & \frac{1}{h_w} \\
                            \end{pmatrix}
                        \] \\ 
    \hline
    Pixels & p & px & \[    M_{s/p} = 
                            \begin{pmatrix}
                            CS_x & 0  \\
                            0 & CS_y \\
                            \end{pmatrix}
                        \] \\ 
    \hline
    Millimeters & m & mm & \[   M_{p/m} = 
                                \begin{pmatrix}
                                (\frac{dm_x}{dp})_x & 0  \\
                                0 & (\frac{dm}{dp})_y \\
                                \end{pmatrix}
                            \]  \\ 
    \hline
\end{tabular}
\end{center}





\end{document}

