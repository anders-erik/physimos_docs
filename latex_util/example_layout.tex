


\documentclass[12pt, letterpaper]{article}

\usepackage{subcaption}             % Subfigure
\usepackage{float}                  % 'H' positioning
\usepackage{graphicx}               % LaTeX package to import graphics

% \numberwithin{equation}{section}


\title{Latex Example -- Document Layout}
\author{Anders Erik}
\date{\today}

\begin{document}



\maketitle


\section{Section 1}

\footnote [1] { footnote tex 1 } 
\footnote [4] { footnote text 2 } 

\section{Another section}

% \includegraphics[width=0.5\textwidth, right]{./tex/chinstrap.png}

% FIGURE USING FLOAT - [H]
\begin{figure}[H]
% \begin{figure}[htbp]
    \centering
    % \includegraphics[width=0.5\textwidth]{./tex/chinstrap.png}
    \caption{A nice plot.}
    \label{fig:mesh1}
\end{figure}


% SUBFIGURE
\begin{figure}[H]

\begin{subfigure}{0.1\textwidth}
% \includegraphics[width=0.9\linewidth, height=0.1\textheight]{./tex/chinstrap.png} 
\caption{Caption1}
\label{fig:subim1}
\end{subfigure}

\begin{subfigure}{0.4\textwidth}
% \includegraphics[width=0.5\linewidth]{./tex/chinstrap.png}
\caption{Caption 2}
\label{fig:subim2}
\end{subfigure}

\caption{Caption for this figure with two images}
\label{fig:image2}

\end{figure}



\subsection{Tables}


% TABLE
\begin{center}
% \begin{tabular}[t]{ | m{5em} | m{1cm}| m{1cm} | }
% \begin{tabular}[b]{ | m{5em} || m{2cm} | m{1cm} | } 
\begin{tabular}{|| c | c | c ||}
  \hline
  cell1& cell2 & cell3 \\ 
  \hline
  cell4 & cell5 & cell6 \\ 
  \hline
  cell7 \newline cell7  & cell8 & cell9 \\ 
  \hline
\end{tabular}
\end{center}


\section*{Indentation Section}

\setlength{\parindent}{20pt}

20pt paragraph indent fdasj dklasjdlk asjdljasldjas jdaskl jdlask jdaslkd jasldasj

20pt indent again

\setlength{\parindent}{0pt}

no indent



This is text contained in the first paragraph. 
This is text contained in the first paragraph. 
This is text contained in the first paragraph.\par
This is text contained in the second paragraph. 
This is text contained in the second paragraph.
This is text contained in the second paragraph.


\subsection*{Alignment}

\begin{center}
Example 1: The following paragraph (given in quotes) is an 
example of centred alignment using the center environment. 

``La\TeX{} is a document preparation system and document markup 
language."
\end{center}


% \ref{fig:your-label}

Hello world!
$Hello world!$ %math mode 


\begin{flushleft}
Flushleft
\end{flushleft}

\begin{flushright}
Flushright
\end{flushright}

Some of the \textbf{greatest}
discoveries in \underline{science} 
were made by \textbf{\textit{accident}}.



\subsection*{Horizontal spacing}

$a \,\,\, + \,\,\, b \:\:\:  + \:\:\: c \;\;\; + \;\;\; d $


\end{document}
