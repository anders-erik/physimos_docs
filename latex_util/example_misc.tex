


\documentclass[12pt, letterpaper]{article}

% \usepackage[export]{adjustbox}    % graphics positioning
\usepackage{array}                  % Table/tabular
\usepackage{subcaption}             % Subfigure
\usepackage{amsmath}                % matrix
\usepackage{float}                  % 'H' positioning
\usepackage{graphicx}               % LaTeX package to import graphics
% \graphicspath{{images}} %configuring the graphicx package
\graphicspath{{.}} %configuring the graphicx package


\numberwithin{equation}{section}


\title{Latex Examples}
\author{Anders Erik}
\date{\today}

\begin{document}

\maketitle

\footnote [1] { footnote tex 1 } 
\footnote [4] { footnote text 2 } 

% \includegraphics[width=0.5\textwidth, right]{./tex/chinstrap.png}

% FIGURE USING FLOAT - [H]
\begin{figure}[H]
% \begin{figure}[htbp]
    \centering
    % \includegraphics[width=0.5\textwidth]{./tex/chinstrap.png}
    \caption{A nice plot.}
    \label{fig:mesh1}
\end{figure}


% SUBFIGURE
\begin{figure}[H]

\begin{subfigure}{0.1\textwidth}
% \includegraphics[width=0.9\linewidth, height=0.1\textheight]{./tex/chinstrap.png} 
\caption{Caption1}
\label{fig:subim1}
\end{subfigure}

\begin{subfigure}{0.4\textwidth}
% \includegraphics[width=0.5\linewidth]{./tex/chinstrap.png}
\caption{Caption 2}
\label{fig:subim2}
\end{subfigure}

\caption{Caption for this figure with two images}
\label{fig:image2}

\end{figure}

% TABLE
\begin{center}
% \begin{tabular}[t]{ | m{5em} | m{1cm}| m{1cm} | }
% \begin{tabular}[b]{ | m{5em} || m{2cm} | m{1cm} | } 
\begin{tabular}{|| c | c | c ||}
  \hline
  cell1& cell2 & cell3 \\ 
  \hline
  cell4 & cell5 & cell6 \\ 
  \hline
  cell7 \newline cell7  & cell8 & cell9 \\ 
  \hline
\end{tabular}
\end{center}


\section*{Indentation}
\setlength{\parindent}{20pt}

20pt paragraph indent fdasj dklasjdlk asjdljasldjas jdaskl jdlask jdaslkd jasldasj

20pt indent again
\setlength{\parindent}{0pt}

no indent



\[
\frac{12}{12}
\]

\[
\pi
\eqno 2.1
\]

\begin{equation}
E = mc^2 \tag{3.1} \label{eq:3.1}
\end{equation}

\begin{equation}
E = mc^2 \label{eq:4.1}
\end{equation}

\begin{equation}
E = mc^2 \tag{5.1}
\end{equation}


$$
\begin{bmatrix}
1 \\% 1 & 2 & 3\\
a% a & b & c
\end{bmatrix}
\eqno 2.2
$$

\[
\begin{pmatrix}
  1 & 2 \\
  3 & 4
\end{pmatrix}
\leqno 2.2
\]

This is text contained in the first paragraph. 
This is text contained in the first paragraph. 
This is text contained in the first paragraph.\par
This is text contained in the second paragraph. 
This is text contained in the second paragraph.
This is text contained in the second paragraph.


\begin{center}
Example 1: The following paragraph (given in quotes) is an 
example of centred alignment using the center environment. 

``La\TeX{} is a document preparation system and document markup 
language. \LaTeX{} uses the \TeX{} typesetting program for formatting 
its output, and is itself written in the \TeX{} macro language. 
\LaTeX{} is not the name of a particular (executable) typesetting program, but 
refers to the suite of commands (\TeX{} macros) which form the markup 
conventions used to typeset \LaTeX{} documents."
\end{center}


% \ref{fig:your-label}

Hello world!
$Hello world!$ %math mode 


\begin{flushleft}
Flushleft
\end{flushleft}

\begin{flushright}
Flushright
\end{flushright}

Some of the \textbf{greatest}
discoveries in \underline{science} 
were made by \textbf{\textit{accident}}.


\section*{Notes for My Paper}

\subsection*{How to handle topicalization}
% \sas
aa sa

$\pi$

$$\pi$$

\[
\pi r^2
\]


\section*{Math}

\[
\alpha \phi \psi \Gamma A
\frac{ \frac{b}{B} }{a}
\def\sum_ {2+2}
\sum_ { \frac{a}{i} + a}
\frac{a}{i}
a \over b  % splits the whole math expression
{a \over b}  % but not this one
\eqno {2}
\]

% % \matrix {a\,b \choose c\,d} % matex

\[
\langle \rho, \Theta, \Phi \rangle
\]

\[
({\Gamma \over dx })''
\]

\[
y_i^3
y^3_i
\]

\[
\cup \bigcup \cap
\]

\[
\begin {pmatrix} a & b \\ c & d \end {pmatrix}
\begin {array}{rcl} a \\ b \\ c  \end {array}
\begin {bmatrix} a \\ b \\ c  \end {bmatrix}
\]

\[
% \mathbb {RAD}
% \mathscr{P}
\infty
\int_{-2}^{2} \prod \coprod
\]

Horizontal spacing $a \,\,\, + \,\,\, b \:\:\:  + \:\:\: c \;\;\; + \;\;\; d $

% $\mathfrak{g}$

Here are some greek letters: $\mu\nu\omega\Omega\upsilon\Upsilon$

\end{document}
